Esta terceira etapa do projeto ``Attribute War'' acrescenta uma camada de blockchain ao \emph{cluster} distribu\'{\i}do, eliminando a lacuna de cust\'odia e auditoria observada no PBL anterior. A economia de pacotes e a troca de cartas passam a exigir transa\c{c}\~oes confirmadas, evitando que a abertura de pacotes, a duplica\c{c}\~ao de cartas ou a fraude em trocas dependam de confian\c{c}a no servidor anfitri\~ao. O objetivo desta entrega \'e demonstrar como a cadeia privada mant\'em escassez verific\'avel, rastreabilidade de propriet\'arios e registro imut\'avel de resultados sem quebrar a experi\^encia em tempo real.

A infraestrutura mant\'em tr\^es servidores de jogo em Go orquestrados com Docker e adiciona um n\'o Ethereum (Geth) e o contrato \emph{PackRegistry} compilado via Hardhat/abigen. Cada servidor conecta-se ao n\'o por vari\'aveis de ambiente (\texttt{BLOCKCHAIN\_NODE\_URL}, \texttt{CONTRACT\_ADDRESS}, \texttt{ADMIN\_PRIVATE\_KEY}), preserva o token ring para o estoque global de cartas em partidas e continua a usar \emph{matchmaking} RESTful (\texttt{/api/find-opponent}, \texttt{/api/request-match}) e o barramento \emph{publish/subscribe} para clientes TCP. O contrato exp\~oe opera\c{c}\~oes de estoque (\texttt{getStock}, \texttt{decrementStock}), posse (\texttt{assignCards}, \texttt{transferCard}) e trilha de partidas (\texttt{recordMatch}), consumidas pelo cliente Go \texttt{blockchain.Client} com espera ativa de recibos para garantir finaliza\c{c}\~ao.

O fluxo de abertura de pacotes debita o estoque \emph{on-chain}, gera cartas base via \texttt{PackSystem}, anexa um UUID a cada carta e registra a posse antes da entrega ao cliente (\texttt{PACK\_OPENED}). A troca de cartas (\texttt{/trade}) executa \texttt{transferCard} e devolve confirma\c{c}\~oes tanto ao remetente quanto ao destinat\'ario se estiver online; a cole\c{c}\~ao (\texttt{/collection}) consulta \texttt{getUserCards} para exibir NFTs j\'a registrados. Resultados de partidas invocam \texttt{recordMatch} em \emph{goroutine}, mantendo o jogo responsivo enquanto a auditoria \'e selada no \emph{ledger}. A su\'{\i}te de testes reutiliza cen\'arios de concorr\^encia e estresse (\texttt{packs\_test.go}, \texttt{stress\_cluster\_test.go}) para verificar controle de estoque e continuidade do \emph{cluster}, agora subordinados ao modelo de confirma\c{c}\~ao de transa\c{c}\~oes.

A \textbf{Figura~1} ilustra a topologia com o anel de servidores, o n\'o Ethereum e o contrato \emph{PackRegistry}, destacando os fluxos REST, TCP e chamadas \emph{on-chain}.

Para orientar o leitor, a Se\c{c}\~ao~2 revisita fundamentos de blockchain aplicados a jogos, contratos Ethereum e consist\^encia distribu\'{\i}da. A Se\c{c}\~ao~3 descreve a metodologia e a implementa\c{c}\~ao do cliente Go, integra\c{c}\~ao com o n\'o Geth e ajustes no protocolo de jogo. A Se\c{c}\~ao~4 discute os resultados experimentais e implica\c{c}\~oes de seguran\c{c}a. A Se\c{c}\~ao~5 encerra com conclus\~oes e extens\~oes futuras, como reconciliar token ring e \emph{ledger} p\'ublico para \emph{marketplaces} externos.
