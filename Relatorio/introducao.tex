Esta segunda etapa do projeto \emph{Attribute War} reestrutura o protótipo centralizado em um cluster distribuído, capaz de atender a uma base de jogadores maior sem comprometer a disponibilidade. Considerando os limites identificados no PBL anterior — gargalo no servidor único, latência crescente e ausência de tolerância a falhas —, uma nova versão foi projetada com múltiplos servidores cooperando em contêineres Docker e comunicação totalmente baseada em APIs. O objetivo desta entrega é demonstrar como a arquitetura distribuída mantém a experiência competitiva em tempo real, preservando a justiça na distribuição de cartas e possibilitando novas dinâmicas sociais entre jogadores.

A infraestrutura integra matchmaking distribuído, API REST \emph{server-to-server} e um barramento \emph{publish-subscribe} para clientes TCP. Cada servidor realiza ciclos de token ring: recebe o token com o estoque global de cartas, analisa sua fila local, cria partidas locais ou entre servidores e, em seguida, repassa o token ao próximo nó. Endpoints como \texttt{/api/find-opponent}, \texttt{/matches/\{id\}/action} e \texttt{/api/request-cards} sustentam a sincronização de estado e a troca de mensagens de jogo; cada mensagem JSON segue o protocolo tipado definido previamente, agora enriquecido com roteamento inteligente para oponente remoto. Para partidas cruzadas, a sincronização de mãos, a resolução de rodadas e as notificações de resultado ocorrem em tempo real, com retransmissão HTTP e entrega via broker local.

O mecanismo de pacotes também foi distribuído: o token transporta o \emph{pool} embaralhado, controla reabastecimentos e registra auditoria de aberturas, evitando duplicidade mesmo sob concorrência elevada. Testes automatizados de concorrência e estresse (\texttt{packs\_test.go}, \texttt{stress\_cluster\_test.go}) validam que o estoque permanece consistente; cenários de falha simulam perda de servidores e confirmam a recuperação por \emph{watchdog} do token e encerramento gracioso de partidas. A \textbf{Figura~1} deve ilustrar a topologia em anel com canais REST e fluxo \emph{publish-subscribe}; a \textbf{Figura~2} pode detalhar o diagrama de sequência de uma partida entre servidores; a \textbf{Figura~3} pode apresentar o pipeline de gestão de cartas e as evidências experimentais (logs, gráficos de RTT, tabelas de estoque).

Para orientar o leitor, a Seção~2 revisita os fundamentos de sistemas distribuídos relevantes para a solução, com ênfase no modelo token ring, em consistência eventual e em protocolos RESTful. A Seção~3 descreve a arquitetura e a implementação, abordando o ciclo do token, os novos endpoints, o broker e a orquestração de partidas remotas. A Seção~4 discute os experimentos sobre escalabilidade, latência e tolerância a falhas. Por fim, a Seção~5 sintetiza as contribuições e aponta extensões planejadas, como a ampliação das mecânicas de interação entre jogadores e a integração de monitoramento observável ao cluster.
