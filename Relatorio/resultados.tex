Esta se\c{c}\~ao apresenta as evid\^encias coletadas nas campanhas de testes descritas na Metodologia, cobrindo integra\c{c}\~ao com o contrato \emph{PackRegistry}, consist\^encia do token ring nas m\~aos iniciais, opera\c{c}\~ao das APIs REST para partidas distribu\'{\i}das, lat\^encia observada e resili\^encia a falhas. Todos os experimentos foram executados no ambiente orquestrado por Docker Compose, garantindo condi\c{c}\~oes controladas e reproduz\'{\i}veis~\cite{docker-compose-ref}.

\subsection*{Liquida\c{c}\~ao on-chain e controle de estoque}
O fluxo \texttt{OPEN\_PACK} foi exercitado em dezenas de requisi\c{c}\~oes simult\^aneas. Cada pedido resultou em (i) chamada \texttt{decrementStock} no contrato, (ii) espera pelo recibo via \texttt{WaitForTransactionReceipt}, (iii) gera\c{c}\~ao de cartas base e anexa\c{c}\~ao de UUIDs, (iv) chamada \texttt{assignCards} e (v) entrega da mensagem \texttt{PACK\_OPENED}. Os logs \texttt{[BLOCKCHAIN]} mostraram hashes distintos para d\'ebito e atribui\c{c}\~ao, e \texttt{[HANDLER]} reportou o estoque lido em \texttt{GetStock()} ap\'os cada confirma\c{c}\~ao. Nenhum pacote foi entregue sem recibo minerado, e n\~ao houve respostas \texttt{OUT\_OF\_STOCK} enquanto o contrato mantinha saldo positivo. A Tabela~\ref{tab:pack-onchain} resume os indicadores principais.

\begin{table}[htbp]
  \centering
  \caption{Indicadores do teste de abertura de pacotes com liquida\c{c}\~ao on-chain.}
  \label{tab:pack-onchain}
  \begin{tabular}{@{}lcc@{}}
    \toprule
    \textbf{Indicador} & \textbf{Valor observado} & \textbf{Efeito} \\
    \midrule
    Requisi\c{c}\~oes concorrentes & 20 goroutines & Exercitou confirma\c{c}\~oes de transa\c{c}\~ao \\
    Estoque inicial (contrato) & 10 pacotes & Recurso escasso controlado pela EVM \\
    Pacotes liberados & 10 (100\% do estoque) & Nenhum d\'ebito sem recibo minerado \\
    Falhas controladas & 10 \texttt{OUT\_OF\_STOCK} & Sem erros inesperados de RPC \\
    Entradas de auditoria & 10 registros de assignCards & Rastreabilidade de dono por UUID \\
    Estoque final (contrato) & 0 & Sem saldo negativo \\
    \bottomrule
  \end{tabular}
\end{table}

Nos testes de integra\c{c}\~ao do cluster, cada servidor consumiu cartas iniciais somente quando detinha o token, como evidenciado por mensagens \texttt{[MATCHMAKING] Pegou 10 cartas do token para a partida} seguidas de \texttt{[MATCHMAKING] A passar o token ...}. A reintrodu\c{c}\~ao autom\'atica (\texttt{refillPool\_unsafe}) foi acionada quando o pool caiu abaixo do limiar, emitindo \texttt{[TOKEN] Pool reabastecido}. Al\'em disso, o registro \texttt{recordMatch} foi disparado ao final das partidas com vencedores definidos, gravando IDs no \emph{ledger}.

\subsection*{Partidas distribu\'{\i}das e sincroniza\c{c}\~ao de estado}
Jogadores conectados a \texttt{server-1} e \texttt{server-2} foram pareados via \texttt{find-opponent}/\texttt{request-match}. O lote de dez cartas retirado do token foi particionado em m\~aos sim\'etricas para anfitri\~ao e convidado, confirmado por logs \texttt{[MATCH]} em ambos os n\'os. Cada jogada remota percorreu \texttt{POST /matches/\{id\}/action}; \texttt{forwardPlayIfNeeded()} evitou eco e manteve determinismo. Os resultados de rodada (\texttt{ROUND\_RESULT}) exibiram danos e HP id\^enticos, e \texttt{MATCH\_END} foi entregue simultaneamente aos dois clientes.

\subsection*{Observabilidade de lat\^encia e vaz\~ao}
O cliente CLI mediu o RTT com \texttt{PING/PONG} peri\'odicos. Durante partidas locais e distribu\'{\i}das, os valores permaneceram em poucos milissegundos, coerentes com rede local e sem perdas de \texttt{PING}. Os logs \texttt{[MATCH]} e \texttt{[MATCHMAKING]} indicaram vaz\~ao sustentada de mensagens sem filas internas, mesmo durante confirma\c{c}\~ao de transa\c{c}\~oes, pois estas s\~ao aguardadas em \emph{goroutine} separada.

\subsection*{Toler\^ancia a falhas e reabastecimento}
Ao interromper deliberadamente um dos tr\^es servidores durante \texttt{tests/stress\_cluster\_test.go}, os demais n\'os continuaram a responder: pacotes foram abertos at\'e o estoque on-chain zerar, e as propor\c{c}\~oes \texttt{PACK\_OPENED}/\texttt{OUT\_OF\_STOCK} corresponderam ao saldo do contrato. O \emph{watchdog} de token promoveu o pr\'oximo n\'o vivo e reconfigurou o anel, evitando travamentos. O pipeline completo abrange leitura do \texttt{cards.json}, rota do token, chamadas \emph{on-chain} de estoque e transfer\^encias, e auditoria de partidas.

Em s\'{\i}ntese, o sistema apresentou:
\begin{itemize}
  \item \textbf{Escassez verific\'avel:} nenhum pacote foi emitido sem confirma\c{c}\~ao on-chain, e o estoque final coincidiu com o registrado no contrato.
  \item \textbf{Consist\^encia interservidores:} partidas distribu\'{\i}das mantiveram HP, m\~aos e resultados id\^enticos entre anfitri\~ao e convidado.
  \item \textbf{Observabilidade:} logs estruturados e RTT expuseram lat\^encia e hashes de transa\c{c}\~ao para diagn\'ostico.
\end{itemize}

Como ponto de aten\c{c}\~ao, o reabastecimento do token para m\~aos iniciais ainda depende do fallback local \texttt{refillHands()} em partidas longas. Embora n\~ao tenha causado inconsist\^encias, evoluir esse mecanismo para consumos exclusivamente mediados pelo token reduzir\'a o risco de diverg\^encias futuras.
