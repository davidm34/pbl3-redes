\documentclass[12pt]{article}

\usepackage{sbc-template}

\usepackage{graphicx,url}

\usepackage[utf8]{inputenc}
\usepackage[T1]{fontenc}
\usepackage[brazil]{babel}
\usepackage{algpseudocode}
\usepackage{algorithm}
\usepackage{algorithmicx}
\usepackage{listings}
\usepackage{verbatim}
\usepackage{amsmath}
\usepackage[table]{xcolor}
\usepackage{booktabs, datatool}
\usepackage{multirow}
\usepackage{latexsym}
\usepackage{amssymb}
\usepackage{longtable}
\usepackage{booktabs}
\usepackage{tabularx}
\usepackage{ragged2e}
    
\sloppy\begin{tiny}

\end{tiny}

\title{Jogo de Cartas Multiplayer Distribuído}

\author{
  \inst{1} Cláudio Daniel Figueredo Peruna \and
  \inst{1} David Neves Dias
}

\address{Departamento de Tecnologia -- Universidade Estadual de Feira de Santana (UEFS) \\ 44036--900 -- Feira de Santana -- Bahia  
\email{danielperuna2012@gmail.com} \email{davidneves852@gmail.com}}

\begin{document} 

\maketitle

\begin{abstract}
This report describes the second delivery of \emph{Attribute War}, in which the previous single-server prototype was re-engineered into a distributed cluster of Go services orchestrated with Docker. The solution combines a token ring for the global card inventory, RESTful matchmaking between servers, and a publish/subscribe bus for TCP clients. Automated stress tests, distributed match simulations, and latency probes demonstrated consistent gameplay, fault recovery, and real-time observability.

\vspace{\baselineskip}

\noindent
\textbf{Keywords:} Distributed systems, Token ring, Multiplayer games, Go, Fault tolerance.
\end{abstract}

\begin{resumo}
Este relat\'orio apresenta a segunda entrega do ``Attribute War'', na qual o prot\'otipo centralizado foi reengenheirado em um cluster distribu\'{\i}do de servi\c{c}os em Go orquestrado com Docker. A solu\c{c}\~ao combina um token ring para o estoque global de cartas, matchmaking RESTful entre servidores e um barramento \emph{publish/subscribe} para clientes TCP. Testes automatizados de estresse, partidas distribu\'{\i}das simuladas e sondas de lat\^encia evidenciaram jogabilidade consistente, recupera\c{c}\~ao diante de falhas e observabilidade em tempo real.

\vspace{\baselineskip}

\noindent
\textbf{Palavras-chave:} Sistemas distribu\'{\i}dos, Token ring, Jogos multiplayer, Go, Toler\^ancia a falhas.

\end{resumo} 

\section{Introdu\c{c}\~ao}
\label{sec:introducao}
Esta terceira etapa do projeto ``Attribute War'' acrescenta uma camada de blockchain ao \emph{cluster} distribu\'{\i}do, eliminando a lacuna de cust\'odia e auditoria observada no PBL anterior. A economia de pacotes e a troca de cartas passam a exigir transa\c{c}\~oes confirmadas, evitando que a abertura de pacotes, a duplica\c{c}\~ao de cartas ou a fraude em trocas dependam de confian\c{c}a no servidor anfitri\~ao. O objetivo desta entrega \'e demonstrar como a cadeia privada mant\'em escassez verific\'avel, rastreabilidade de propriet\'arios e registro imut\'avel de resultados sem quebrar a experi\^encia em tempo real.

A infraestrutura mant\'em tr\^es servidores de jogo em Go orquestrados com Docker e adiciona um n\'o Ethereum (Geth) e o contrato \emph{PackRegistry} compilado via Hardhat/abigen. Cada servidor conecta-se ao n\'o por vari\'aveis de ambiente (\texttt{BLOCKCHAIN\_NODE\_URL}, \texttt{CONTRACT\_ADDRESS}, \texttt{ADMIN\_PRIVATE\_KEY}), preserva o token ring para o estoque global de cartas em partidas e continua a usar \emph{matchmaking} RESTful (\texttt{/api/find-opponent}, \texttt{/api/request-match}) e o barramento \emph{publish/subscribe} para clientes TCP. O contrato exp\~oe opera\c{c}\~oes de estoque (\texttt{getStock}, \texttt{decrementStock}), posse (\texttt{assignCards}, \texttt{transferCard}) e trilha de partidas (\texttt{recordMatch}), consumidas pelo cliente Go \texttt{blockchain.Client} com espera ativa de recibos para garantir finaliza\c{c}\~ao.

O fluxo de abertura de pacotes debita o estoque \emph{on-chain}, gera cartas base via \texttt{PackSystem}, anexa um UUID a cada carta e registra a posse antes da entrega ao cliente (\texttt{PACK\_OPENED}). A troca de cartas (\texttt{/trade}) executa \texttt{transferCard} e devolve confirma\c{c}\~oes tanto ao remetente quanto ao destinat\'ario se estiver online; a cole\c{c}\~ao (\texttt{/collection}) consulta \texttt{getUserCards} para exibir NFTs j\'a registrados. Resultados de partidas invocam \texttt{recordMatch} em \emph{goroutine}, mantendo o jogo responsivo enquanto a auditoria \'e selada no \emph{ledger}. A su\'{\i}te de testes reutiliza cen\'arios de concorr\^encia e estresse (\texttt{packs\_test.go}, \texttt{stress\_cluster\_test.go}) para verificar controle de estoque e continuidade do \emph{cluster}, agora subordinados ao modelo de confirma\c{c}\~ao de transa\c{c}\~oes.

A \textbf{Figura~1} ilustra a topologia com o anel de servidores, o n\'o Ethereum e o contrato \emph{PackRegistry}, destacando os fluxos REST, TCP e chamadas \emph{on-chain}.

Para orientar o leitor, a Se\c{c}\~ao~2 revisita fundamentos de blockchain aplicados a jogos, contratos Ethereum e consist\^encia distribu\'{\i}da. A Se\c{c}\~ao~3 descreve a metodologia e a implementa\c{c}\~ao do cliente Go, integra\c{c}\~ao com o n\'o Geth e ajustes no protocolo de jogo. A Se\c{c}\~ao~4 discute os resultados experimentais e implica\c{c}\~oes de seguran\c{c}a. A Se\c{c}\~ao~5 encerra com conclus\~oes e extens\~oes futuras, como reconciliar token ring e \emph{ledger} p\'ublico para \emph{marketplaces} externos.


\section{Fundamenta\c{c}\~ao Te\'orica}
\label{sec:fundamentos}
A evolu\c{c}\~ao de arquiteturas para jogos massivos multijogador (MMO) tem migrado de modelos puramente centralizados para ecossistemas distribu\'{\i}dos que priorizam a propriedade de ativos digitais e a transpar\^encia das regras de neg\'ocio. O projeto atual adota uma abordagem h\'{\i}brida, combinando a performance de servidores de jogo distribu\'{\i}dos com a seguran\c{c}a e imutabilidade de uma rede \emph{Blockchain} para o gerenciamento de estados cr\'{\i}ticos.

\subsection{Blockchain e Livros-Raz\~ao Distribu\'{\i}dos}
Diferente do prot\'otipo anterior, que dependia de um anel l\'ogico de servidores para consist\^encia, a nova solu\c{c}\~ao utiliza uma \emph{Blockchain} como camada base de verdade. Uma \emph{blockchain} \'e, essencialmente, um livro-raz\~ao distribu\'{\i}do e anexado sequencialmente (\emph{append-only}), onde transa\c{c}\~oes s\~ao validadas por consenso entre n\'os participantes, eliminando a necessidade de uma autoridade central confi\'avel~\cite{zheng2017}.

No contexto do projeto, a rede Ethereum \'e utilizada para garantir que o ``estoque global de pacotes'' e a propriedade das cartas sejam geridos de forma descentralizada. A arquitetura da Ethereum implementa uma m\'aquina de estados quase-Turing-completa (\emph{EVM - Ethereum Virtual Machine}), permitindo n\~ao apenas o registro de transfer\^encias de valor, mas a execu\c{c}\~ao de c\'odigo arbitr\'ario replicado em todos os n\'os da rede~\cite{wood2014}. Isso resolve nativamente o problema de \emph{double-spending} (ou dupla abertura de pacotes) sem a complexidade de algoritmos de elei\c{c}\~ao de l\'{\i}der ou recupera\c{c}\~ao de tokens perdidos.

\subsection{Smart Contracts como Regras de Neg\'ocio}
A l\'ogica de distribui\c{c}\~ao de cartas e registro de partidas foi encapsulada em \textbf{Smart Contracts} escritos na linguagem Solidity. O conceito de contratos inteligentes, introduzido teoricamente por Nick Szabo, refere-se a protocolos computacionais que facilitam, verificam ou imp\~oem a negocia\c{c}\~ao ou execu\c{c}\~ao de um contrato digitalmente~\cite{szabo1997}.

O contrato \texttt{PackRegistry.sol} atua como um terceiro confi\'avel e aut\^onomo. Ao inv\'es de confiar que um servidor de jogo respeite o limite de estoque, o c\'odigo imut\'avel na \emph{blockchain} rejeita matematicamente qualquer transa\c{c}\~ao que tente decrementar o estoque abaixo de zero ou transferir uma carta sem a assinatura criptogr\'afica do propriet\'ario. Isso garante integridade forte (\emph{strong consistency}) para ativos digitais, enquanto libera os servidores de jogo para lidarem com a l\'ogica r\'apida e vol\'atil das partidas~\cite{antonopoulos2018}.

\subsection{Arquitetura H\'{\i}brida: On-chain e Off-chain}
Sistemas de jogos baseados em \emph{blockchain} frequentemente enfrentam o desafio da escalabilidade e lat\^encia. Para mitigar isso, o projeto adota um padr\~ao de arquitetura h\'{\i}brida:
\begin{itemize}
    \item \textbf{Off-chain (Alta Velocidade):} A comunica\c{c}\~ao em tempo real durante as batalhas (troca de movimentos, c\'alculo de dano) ocorre via conex\~oes TCP diretas e APIs REST entre servidores Go, mantendo a baixa lat\^encia necess\'aria para a experi\^encia do usu\'ario~\cite{fielding2000}.
    \item \textbf{On-chain (Alta Seguran\c{c}a):} Eventos cr\'{\i}ticos, como a abertura de um pacote de cartas, o resultado final de uma partida e a troca de cartas entre jogadores, s\~ao submetidos \`a \emph{blockchain} via transa\c{c}\~oes assinadas.
\end{itemize}

Essa separa\c{c}\~ao permite que o sistema escale horizontalmente o \emph{matchmaking} (via \emph{publish-subscribe} e orquestra\c{c}\~ao de cont\^eineres Docker), enquanto mant\'em um registro audit\'avel e persistente dos ativos no livro-raz\~ao compartilhado~\cite{xu2019}.

\subsection{Integ\-ra\c{c}\~ao e Ferramentas}
A interoperabilidade entre o ambiente de execu\c{c}\~ao \texttt{Go} e a \emph{Ethereum Virtual Machine} \'e viabilizada pelo uso de \emph{bindings} gerados (via \texttt{abigen}) e pelo protocolo JSON-RPC. O cliente Go atua como um n\'o leve ou interage com um n\'o Geth (\emph{Go-Ethereum}), assinando transa\c{c}\~oes com chaves privadas geridas localmente. A orquestra\c{c}\~ao de todo esse ambiente --- incluindo n\'os de \emph{blockchain} privados, servidores de aplica\c{c}\~ao e bancos de dados --- continua sendo gerida via Docker Compose, essencial para garantir a reprodutibilidade dos testes de consenso e minera\c{c}\~ao em ambiente laboratorial.

A Tabela~\ref{tab:fund-teorica-pilares-new} atualiza os pilares te\'oricos do projeto, refletindo a substitui\c{c}\~ao do anel de tokens pela tecnologia de registros distribu\'{\i}dos.

\begin{table}[htbp]
  \centering
  \caption{Pilares te\'oricos atualizados para a arquitetura baseada em Blockchain.}
  \label{tab:fund-teorica-pilares-new}
  \begin{tabular}{@{} p{0.22\linewidth} p{0.34\linewidth} p{0.34\linewidth} @{}}
    \toprule
    \textbf{Pilar} & \textbf{Refer\^encia Principal} & \textbf{Aplica\c{c}\~ao no Projeto} \\
    \midrule
    Blockchain Architecture & Wood~\cite{wood2014}, Zheng \emph{et al.}~\cite{zheng2017} & Livro-raz\~ao imut\'avel para persist\^encia do estado global (estoque e invent\'ario). \\
    Smart Contracts & Szabo~\cite{szabo1997}, Antonopoulos~\cite{antonopoulos2018} & L\'ogica aut\^onoma para distribui\c{c}\~ao de pacotes e valida\c{c}\~ao de trocas (\texttt{PackRegistry}). \\
    Arquitetura H\'{\i}brida & Xu \emph{et al.}~\cite{xu2019} & Separa\c{c}\~ao entre jogabilidade (Off-chain) e liquida\c{c}\~ao de ativos (On-chain). \\
    RESTful APIs & Fielding~\cite{fielding2000} & Comunica\c{c}\~ao s\'incrona entre servidores para \emph{matchmaking}. \\
    Publish-Subscribe & Eugster \emph{et al.}~\cite{eugster2003} & Desacoplamento de mensagens de eventos para clientes conectados. \\
    Concorr\^encia em Go & The Go Authors~\cite{go-tour-concurrency} & Gerenciamento de m\'ultiplas conex\~oes TCP e \emph{listeners} de eventos da \emph{chain}. \\
    \bottomrule
  \end{tabular}
\end{table}

A \textbf{Figura~\ref{fig:blockchain-arch}} (Figura~1) ilustra a topologia consolidada: o anel de servidores de jogo conecta-se ao n\'o Ethereum, executa o token ring para m\~aos iniciais e submete transa\c{c}\~oes ao contrato \emph{PackRegistry}.

\begin{figure}[ht]
  \centering
  \includegraphics[width=0.95\linewidth]{tec502-modelo-latex-relatorio-pbl/figuras/arquitetura.png}
  \caption{Topologia distribu\'{\i}da com token ring e integra\c{c}\~ao ao contrato \emph{PackRegistry}.}
  \label{fig:blockchain-arch}
\end{figure}
\FloatBarrier


\section{Metodologia, Implementa\c{c}\~ao e Testes}
\label{sec:metodologia}
A metodologia adotada para o Problema~3 concentrou-se em incorporar uma camada de blockchain ao \emph{cluster} distribu\'{\i}do do \emph{Attribute War}, garantindo escassez verific\'avel de pacotes e rastreabilidade das cartas sem comprometer o desempenho das partidas. O processo abrangeu o provisionamento conjunto do anel de servidores e do n\'o Ethereum, o ciclo do token ring para m\~aos iniciais, a integra\c{c}\~ao das APIs interservidores, o roteamento \emph{publish/subscribe} para clientes TCP e a valida\c{c}\~ao automatizada de cen\'arios de concorr\^encia e de liquida\c{c}\~ao \emph{on-chain}.

\subsection*{Provisionamento do cluster e par\^ametros operacionais}
O ciclo inicia com a orquestra\c{c}\~ao declarativa em \texttt{docker-compose.yml}: tr\^es servi\c{c}os \texttt{server-n}, um \texttt{client} e o \texttt{blockchain-node} compartilham a rede \texttt{game-net}. Vari\'aveis de ambiente de cada servidor (\texttt{LISTEN\_ADDR}, \texttt{API\_ADDR}, \texttt{ALL\_SERVERS}, \texttt{BLOCKCHAIN\_NODE\_URL}, \texttt{CONTRACT\_ADDRESS}, \texttt{ADMIN\_PRIVATE\_KEY}) definem portas, topologia do anel e credenciais de acesso ao contrato \emph{PackRegistry}, preservando reprodutibilidade do ambiente~\cite{docker-compose-ref}. No \emph{bootstrap}, \texttt{server/main.go} determina sua posi\c{c}\~ao no anel, configura o cliente Ethereum (\texttt{blockchain.NewClient}) com checagem de \emph{chainId} e carrega o \emph{State Manager}, o \emph{Broker} e o servi\c{c}o de matchmaking.

\subsection*{Ciclo do token ring, ledger e distribui\c{c}\~ao de cartas}
O estoque usado nas partidas permanece governado por um token que circula em anel, oferecendo exclus\~ao m\'utua e ordem total para retirada de cartas iniciais. O \texttt{MatchmakingService.Run()} bloqueia at\'e receber o token, processa filas locais e distribui m\~aos com \texttt{DrawCards()} para \texttt{CreateLocalMatchWithCards()} ou \texttt{CreateDistributedMatchAsHostWithCards()}. O l\'ider regenera o token a partir de \texttt{cards.json} quando o \emph{watchdog} expira, evitando esgotamento. Em paralelo, a escassez global de pacotes \'e delegada ao contrato \emph{PackRegistry}, que rejeita decr\'escimos indevidos e modela a propriedade como estado imut\'avel~\cite{wood2014, szabo1997}.

\subsection*{Matchmaking distribu\'{\i}do e API interservidores}
A coopera\c{c}\~ao entre servidores segue o estilo REST com mensagens JSON~\cite{fielding2000}. Quando apenas um jogador est\'a na fila local, o host percorre \texttt{ALL\_SERVERS} via \texttt{GET /api/find-opponent}; em caso de sucesso, envia \texttt{POST /api/request-match} com as cartas do convidado retiradas do token. Jogadas remotas usam \texttt{POST /matches/\{id\}/action}, aplicando \texttt{match.PlayCard()} em ambos os lados; o protocolo impede eco de mensagens ao distinguir anfitri\~ao e convidado. A passagem do token entre n\'os (\texttt{/api/receive-token}) garante que apenas um servidor manipule o \emph{pool} de cartas de partida por vez, enquanto o \emph{ledger} on-chain persiste ativos.

\subsection*{Canal cliente-servidor e roteamento publish-subscribe}
Clientes TCP autenticam-se, assinam os t\'opicos \texttt{player.\textless id\textgreater} no \texttt{pubsub.Broker} e recebem \texttt{MATCH\_FOUND}, \texttt{STATE} e \texttt{ROUND\_RESULT}. O padr\~ao \emph{publish/subscribe} desacopla produtores de consumidores, reduzindo bloqueios entre \emph{goroutines}~\cite{eugster2003}. A abertura de pacotes \texttt{OPEN\_PACK} aciona \texttt{blockchain.DecrementStock()}, espera o recibo (\texttt{WaitForTransactionReceipt}), gera cartas base via \texttt{PackSystem}, anexa UUIDs, registra a posse com \texttt{AssignCards()} e retorna \texttt{PACK\_OPENED} com estoque atualizado. A troca \texttt{/trade} chama \texttt{TransferCard()} e notifica remetente e destinat\'ario; a consulta \texttt{/collection} utiliza \texttt{GetUserCards()} para exibir NFTs j\'a registrados. O \texttt{PING/PONG} segue peri\'odico para medir lat\^encia fim a fim.

\subsection*{Automatiza\c{c}\~ao de testes e monitoramento}
A valida\c{c}\~ao automatizada usa \texttt{go test} em dois eixos. O \texttt{tests/packs\_test.go} modela concorr\^encia de abertura de pacotes com 20 \emph{goroutines}, aferindo atomici\-dade, aus\^encia de duplicatas e exaust\~ao do estoque, com logs de auditoria. O \texttt{tests/stress\_cluster\_test.go} compila o servidor, sobe tr\^es inst\^ancias, injeta centenas de clientes simulados e verifica a propor\c{c}\~ao entre \texttt{PACK\_OPENED} e \texttt{OUT\_OF\_STOCK}, mesmo sob falha intencional de um n\'o. Logs estruturados (\texttt{[BLOCKCHAIN]}, \texttt{[MATCHMAKING]}, \texttt{[MATCH]}) e o RTT medido por \texttt{PING/PONG} auxiliam o diagn\'ostico. A Tabela~\ref{tab:testes-metodologia-blockchain} resume os casos principais e as evid\^encias esperadas.

\begin{table}[htbp]
  \centering
  \caption{Casos de teste e evid\^encias de valida\c{c}\~ao do cluster com blockchain.}
  \label{tab:testes-metodologia-blockchain}
  \begin{tabular}{@{}p{1.2cm} p{3cm} p{4.2cm} p{4.0cm}@{}}
    \toprule
    \textbf{ID} & \textbf{Escopo} & \textbf{Procedimento} & \textbf{Evid\^encia de sucesso} \\
    \midrule
    T1 & Conex\~ao blockchain & Inicializar \texttt{docker-compose up} & Logs \texttt{[BLOCKCHAIN]} exibem \emph{chainId} e estoque inicial lido do contrato. \\
    T2 & Token ring & Receber e repassar token entre os tr\^es servidores & Mensagens \texttt{Token recebido/passo} alternam entre n\'os sem bloqueio. \\
    T3 & Partida distribu\'{\i}da & Conectar clientes em servidores distintos e acionar \texttt{FIND\_MATCH} & Sequ\^encia \texttt{find-opponent} $\rightarrow$ \texttt{request-match} $\rightarrow$ \texttt{MATCH\_FOUND} sim\'etrica; m\~aos id\^enticas em ambos os lados. \\
    T4 & Abertura de pacote \emph{on-chain} & Enviar \texttt{OPEN\_PACK} e aguardar confirma\c{c}\~ao & Decremento de estoque confirmado, UUIDs atribu\'{\i}dos e retorno \texttt{PACK\_OPENED}. \\
    T5 & Troca de cartas & Executar \texttt{/trade} entre dois jogadores & \texttt{TransferCard} aceita, recibo confirmado e mensagem de recebimento para o destinat\'ario. \\
    T6 & Toler\^ancia a falhas & Matar um servidor durante o teste de estresse & Demais n\'os continuam a responder; propor\c{c}\~ao \texttt{PACK\_OPENED}/\texttt{OUT\_OF\_STOCK} coerente com estoque global. \\
    \bottomrule
  \end{tabular}
\end{table}

\subsection*{Fluxo consolidado}
O fluxo consolidado segue os passos:
\begin{enumerate}
  \item Inicializa\c{c}\~ao: cada servidor descobre sua posi\c{c}\~ao no anel, cria clientes Ethereum e valida acesso ao contrato.
  \item Regenera\c{c}\~ao do token: o l\'ider embaralha o \emph{pool} de cartas e injeta o token para iniciar o ciclo.
  \item Matchmaking: prefer\^encia por partidas locais; na falta de oponente, consulta remota e cria partida distribu\'{\i}da com cartas fornecidas pelo token.
  \item Liquida\c{c}\~ao de ativos: pedidos \texttt{OPEN\_PACK} e \texttt{/trade} emitem transa\c{c}\~oes, aguardam recibos e notificam clientes.
  \item Observabilidade: PING/PONG e logs estruturados exp\~oem lat\^encia, consumo de estoque e hashes de transa\c{c}\~ao.
  \item Repassa token: ap\'os processar filas e partidas, o n\'o serializa o token e envia ao pr\'oximo servidor, reiniciando o ciclo.
\end{enumerate}

Essa metodologia alinha a arquitetura distribu\'{\i}da com garantias fortes de posse e de escassez providas pelo \emph{ledger}, preservando a reprodutibilidade e a valida\c{c}\~ao automatizada requeridas para o Problema~3.


\section{Resultados e Discuss\~oes}
\label{sec:resultados}
Esta se\c{c}\~ao apresenta as evid\^encias coletadas nas campanhas de testes descritas na Metodologia, cobrindo integra\c{c}\~ao com o contrato \emph{PackRegistry}, consist\^encia do token ring nas m\~aos iniciais, opera\c{c}\~ao das APIs REST para partidas distribu\'{\i}das, lat\^encia observada e resili\^encia a falhas. Todos os experimentos foram executados no ambiente orquestrado por Docker Compose, garantindo condi\c{c}\~oes controladas e reproduz\'{\i}veis~\cite{docker-compose-ref}.

\subsection*{Liquida\c{c}\~ao on-chain e controle de estoque}
O fluxo \texttt{OPEN\_PACK} foi exercitado em dezenas de requisi\c{c}\~oes simult\^aneas. Cada pedido resultou em (i) chamada \texttt{decrementStock} no contrato, (ii) espera pelo recibo via \texttt{WaitForTransactionReceipt}, (iii) gera\c{c}\~ao de cartas base e anexa\c{c}\~ao de UUIDs, (iv) chamada \texttt{assignCards} e (v) entrega da mensagem \texttt{PACK\_OPENED}. Os logs \texttt{[BLOCKCHAIN]} mostraram hashes distintos para d\'ebito e atribui\c{c}\~ao, e \texttt{[HANDLER]} reportou o estoque lido em \texttt{GetStock()} ap\'os cada confirma\c{c}\~ao. Nenhum pacote foi entregue sem recibo minerado, e n\~ao houve respostas \texttt{OUT\_OF\_STOCK} enquanto o contrato mantinha saldo positivo. A Tabela~\ref{tab:pack-onchain} resume os indicadores principais.

\begin{table}[htbp]
  \centering
  \caption{Indicadores do teste de abertura de pacotes com liquida\c{c}\~ao on-chain.}
  \label{tab:pack-onchain}
  \begin{tabular}{@{}lcc@{}}
    \toprule
    \textbf{Indicador} & \textbf{Valor observado} & \textbf{Efeito} \\
    \midrule
    Requisi\c{c}\~oes concorrentes & 20 goroutines & Exercitou confirma\c{c}\~oes de transa\c{c}\~ao \\
    Estoque inicial (contrato) & 10 pacotes & Recurso escasso controlado pela EVM \\
    Pacotes liberados & 10 (100\% do estoque) & Nenhum d\'ebito sem recibo minerado \\
    Falhas controladas & 10 \texttt{OUT\_OF\_STOCK} & Sem erros inesperados de RPC \\
    Entradas de auditoria & 10 registros de assignCards & Rastreabilidade de dono por UUID \\
    Estoque final (contrato) & 0 & Sem saldo negativo \\
    \bottomrule
  \end{tabular}
\end{table}

Nos testes de integra\c{c}\~ao do cluster, cada servidor consumiu cartas iniciais somente quando detinha o token, como evidenciado por mensagens \texttt{[MATCHMAKING] Pegou 10 cartas do token para a partida} seguidas de \texttt{[MATCHMAKING] A passar o token ...}. A reintrodu\c{c}\~ao autom\'atica (\texttt{refillPool\_unsafe}) foi acionada quando o pool caiu abaixo do limiar, emitindo \texttt{[TOKEN] Pool reabastecido}. Al\'em disso, o registro \texttt{recordMatch} foi disparado ao final das partidas com vencedores definidos, gravando IDs no \emph{ledger}.

\subsection*{Partidas distribu\'{\i}das e sincroniza\c{c}\~ao de estado}
Jogadores conectados a \texttt{server-1} e \texttt{server-2} foram pareados via \texttt{find-opponent}/\texttt{request-match}. O lote de dez cartas retirado do token foi particionado em m\~aos sim\'etricas para anfitri\~ao e convidado, confirmado por logs \texttt{[MATCH]} em ambos os n\'os. Cada jogada remota percorreu \texttt{POST /matches/\{id\}/action}; \texttt{forwardPlayIfNeeded()} evitou eco e manteve determinismo. Os resultados de rodada (\texttt{ROUND\_RESULT}) exibiram danos e HP id\^enticos, e \texttt{MATCH\_END} foi entregue simultaneamente aos dois clientes.

\subsection*{Observabilidade de lat\^encia e vaz\~ao}
O cliente CLI mediu o RTT com \texttt{PING/PONG} peri\'odicos. Durante partidas locais e distribu\'{\i}das, os valores permaneceram em poucos milissegundos, coerentes com rede local e sem perdas de \texttt{PING}. Os logs \texttt{[MATCH]} e \texttt{[MATCHMAKING]} indicaram vaz\~ao sustentada de mensagens sem filas internas, mesmo durante confirma\c{c}\~ao de transa\c{c}\~oes, pois estas s\~ao aguardadas em \emph{goroutine} separada.

\subsection*{Toler\^ancia a falhas e reabastecimento}
Ao interromper deliberadamente um dos tr\^es servidores durante \texttt{tests/stress\_cluster\_test.go}, os demais n\'os continuaram a responder: pacotes foram abertos at\'e o estoque on-chain zerar, e as propor\c{c}\~oes \texttt{PACK\_OPENED}/\texttt{OUT\_OF\_STOCK} corresponderam ao saldo do contrato. O \emph{watchdog} de token promoveu o pr\'oximo n\'o vivo e reconfigurou o anel, evitando travamentos. O pipeline completo abrange leitura do \texttt{cards.json}, rota do token, chamadas \emph{on-chain} de estoque e transfer\^encias, e auditoria de partidas.

Em s\'{\i}ntese, o sistema apresentou:
\begin{itemize}
  \item \textbf{Escassez verific\'avel:} nenhum pacote foi emitido sem confirma\c{c}\~ao on-chain, e o estoque final coincidiu com o registrado no contrato.
  \item \textbf{Consist\^encia interservidores:} partidas distribu\'{\i}das mantiveram HP, m\~aos e resultados id\^enticos entre anfitri\~ao e convidado.
  \item \textbf{Observabilidade:} logs estruturados e RTT expuseram lat\^encia e hashes de transa\c{c}\~ao para diagn\'ostico.
\end{itemize}

Como ponto de aten\c{c}\~ao, o reabastecimento do token para m\~aos iniciais ainda depende do fallback local \texttt{refillHands()} em partidas longas. Embora n\~ao tenha causado inconsist\^encias, evoluir esse mecanismo para consumos exclusivamente mediados pelo token reduzir\'a o risco de diverg\^encias futuras.


\section{Conclus\~ao}
\label{sec:conclusao}
O terceiro ciclo do \emph{Attribute War} comprovou que a incorpora\c{c}\~ao da blockchain ao cluster distribu\'{\i}do tornou verific\'aveis a escassez de pacotes e a posse de cartas, sem degradar a jogabilidade em tempo real. O token ring manteve a coer\^encia das m\~aos iniciais, enquanto o contrato \emph{PackRegistry} arbitrou d\'ebitos e transfer\^encias, eliminando duplica\c{c}\~oes e ancorando auditoria de partidas. A API REST interservidores e o barramento \emph{publish/subscribe} preservaram sincronismo entre anfitri\~ao e convidado; testes automatizados e logs estruturados evidenciaram partidas distribu\'{\i}das consistentes, confirma\c{c}\~oes on-chain e lat\^encia controlada.

Os objetivos foram alcan\c{c}ados: pacotes s\'o foram liberados com recibo minerado, \texttt{recordMatch} registrou vencedores no \emph{ledger}, e o token circulou mesmo sob falhas transit\'orias de n\'os. Persistem limita\c{c}\~oes: o reabastecimento do token para partidas longas ainda depende do \texttt{refillHands()}, o que deixa parte do ciclo de cartas fora do controle estrito do token; al\'em disso, falta telemetria externa (histogramas de RTT, contadores de transa\c{c}\~oes e de partidas) para fechar o ciclo de observabilidade.

Pr\'oximos passos incluem estender o token para fornecer cartas adicionais em rodadas prolongadas, expor m\'etricas de desempenho e de blockchain em pain\'eis agregados e ampliar testes de caos (atraso deliberado na passagem do token, falhas de RPC ou de minera\c{c}\~ao). Essas evolu\c{c}\~oes consolidariam o ecossistema distribu\'{\i}do, elevando seguran\c{c}a e confiabilidade para opera\c{c}\~ao em ambientes multi-regi\~ao.


\bibliographystyle{sbc}
\bibliography{bibliografia}

\end{document}      
