O terceiro ciclo do \emph{Attribute War} comprovou que a incorpora\c{c}\~ao da blockchain ao cluster distribu\'{\i}do tornou verific\'aveis a escassez de pacotes e a posse de cartas, sem degradar a jogabilidade em tempo real. O token ring manteve a coer\^encia das m\~aos iniciais, enquanto o contrato \emph{PackRegistry} arbitrou d\'ebitos e transfer\^encias, eliminando duplica\c{c}\~oes e ancorando auditoria de partidas. A API REST interservidores e o barramento \emph{publish/subscribe} preservaram sincronismo entre anfitri\~ao e convidado; testes automatizados e logs estruturados evidenciaram partidas distribu\'{\i}das consistentes, confirma\c{c}\~oes on-chain e lat\^encia controlada.

Os objetivos foram alcan\c{c}ados: pacotes s\'o foram liberados com recibo minerado, \texttt{recordMatch} registrou vencedores no \emph{ledger}, e o token circulou mesmo sob falhas transit\'orias de n\'os. Persistem limita\c{c}\~oes: o reabastecimento do token para partidas longas ainda depende do \texttt{refillHands()}, o que deixa parte do ciclo de cartas fora do controle estrito do token; al\'em disso, falta telemetria externa (histogramas de RTT, contadores de transa\c{c}\~oes e de partidas) para fechar o ciclo de observabilidade.

Pr\'oximos passos incluem estender o token para fornecer cartas adicionais em rodadas prolongadas, expor m\'etricas de desempenho e de blockchain em pain\'eis agregados e ampliar testes de caos (atraso deliberado na passagem do token, falhas de RPC ou de minera\c{c}\~ao). Essas evolu\c{c}\~oes consolidariam o ecossistema distribu\'{\i}do, elevando seguran\c{c}a e confiabilidade para opera\c{c}\~ao em ambientes multi-regi\~ao.
