O segundo ciclo do \emph{Attribute War} demonstrou que a migra\c{c}\~ao para uma arquitetura distribu\'{\i}da tornou vi\'avel a escalabilidade, a toler\^ancia a falhas e o controle global do estoque de cartas exigidos no enunciado. O token ring concentrou a coer\^encia do recurso compartilhado, enquanto a API REST entre servidores e o barramento \emph{publish-subscribe} asseguraram que partidas cruzadas mantivessem estado id\^entico em ambos os lados. Testes automatizados e logs estruturados evidenciaram que o sistema suporta concorr\^encia intensa sem violar invariantes, preservando tamb\'em a observabilidade de lat\^encia em tempo real.

O projeto alcan\c{c}ou os objetivos propostos: partidas remotas foram sincronizadas, o estoque n\~ao gerou duplicidades e o cluster redistribuiu o token mesmo sob falhas transit\'orias. Como limita\c{c}\~ao remanescente, o reabastecimento de cartas durante partidas longas ainda depende do mecanismo local de \texttt{refillHands()}, o que restringe a completude da distribui\c{c}\~ao; al\'em disso, faltam m\'etricas expostas externamente (histogramas de RTT, contadores de partidas) para fechar o ciclo de observabilidade.

Os pr\'oximos passos naturais incluem estender o token para fornecer cartas adicionais em rodadas prolongadas, endurecer a telemetria com pain\'eis agregados e ampliar o leque de testes de caos (perda de rede, atraso deliberado na passagem do token). Essas evolu\c{c}\~oes consolidariam o ecossistema distribu\'{\i}do e tornariam a aplica\c{c}\~ao apta a hospedar experi\^encias massivas em um ambiente multi-regi\~ao.
