
A evolução de arquiteturas para jogos massivos multijogador (MMO) tem migrado de modelos puramente centralizados para ecossistemas distribuídos que priorizam a propriedade de ativos digitais e a transparência das regras de negócio. O projeto atual adota uma abordagem híbrida, combinando a performance de servidores de jogo distribuídos com a segurança e imutabilidade de uma rede \emph{Blockchain} para o gerenciamento de estados críticos.

\subsection{Blockchain e Livros-Razão Distribuídos}
Diferente do protótipo anterior, que dependia de um anel lógico de servidores para consistência, a nova solução utiliza uma \emph{Blockchain} como camada base de verdade. Uma \emph{blockchain} é, essencialmente, um livro-razão distribuído e anexado sequencialmente (\emph{append-only}), onde transações são validadas por consenso entre nós participantes, eliminando a necessidade de uma autoridade central confiável~\cite{zheng2017}.

No contexto do projeto, a rede Ethereum é utilizada para garantir que o "estoque global de pacotes" e a propriedade das cartas sejam geridos de forma descentralizada. A arquitetura da Ethereum implementa uma máquina de estados quase-Turing-completa (\emph{EVM - Ethereum Virtual Machine}), permitindo não apenas o registro de transferências de valor, mas a execução de código arbitrário replicado em todos os nós da rede~\cite{wood2014}. Isso resolve nativamente o problema de \emph{double-spending} (ou dupla abertura de pacotes) sem a complexidade de algoritmos de eleição de líder ou recuperação de tokens perdidos.

\subsection{Smart Contracts como Regras de Negócio}
A lógica de distribuição de cartas e registro de partidas foi encapsulada em \textbf{Smart Contracts} escritos na linguagem Solidity. O conceito de contratos inteligentes, introduzido teoricamente por Nick Szabo, refere-se a protocolos computacionais que facilitam, verificam ou impõem a negociação ou execução de um contrato digitalmente~\cite{szabo1997}.

O contrato \texttt{PackRegistry.sol} atua como um terceiro confiável e autônomo. Ao invés de confiar que um servidor de jogo respeite o limite de estoque, o código imutável na \emph{blockchain} rejeita matematicamente qualquer transação que tente decrementar o estoque abaixo de zero ou transferir uma carta sem a assinatura criptográfica do proprietário. Isso garante integridade forte (\emph{strong consistency}) para ativos digitais, enquanto libera os servidores de jogo para lidarem com a lógica rápida e volátil das partidas~\cite{antonopoulos2018}.

\subsection{Arquitetura Híbrida: On-chain e Off-chain}
Sistemas de jogos baseados em \emph{blockchain} frequentemente enfrentam o desafio da escalabilidade e latência. Para mitigar isso, o projeto adota um padrão de arquitetura híbrida:
\begin{itemize}
    \item \textbf{Off-chain (Alta Velocidade):} A comunicação em tempo real durante as batalhas (troca de movimentos, cálculo de dano) ocorre via conexões TCP diretas e APIs REST entre servidores Go, mantendo a baixa latência necessária para a experiência do usuário~\cite{fielding2000}.
    \item \textbf{On-chain (Alta Segurança):} Eventos críticos, como a abertura de um pacote de cartas, o resultado final de uma partida e a troca de cartas entre jogadores, são submetidos à \emph{blockchain} via transações assinadas.
\end{itemize}

Essa separação permite que o sistema escale horizontalmente o \emph{matchmaking} (via \emph{publish-subscribe} e orquestração de contêineres Docker), enquanto mantém um registro auditável e persistente dos ativos no livro-razão compartilhado~\cite{xu2019}.

\subsection{Integração e Ferramentas}
A interoperabilidade entre o ambiente de execução \texttt{Go} e a \emph{Ethereum Virtual Machine} é viabilizada pelo uso de \emph{bindings} gerados (via \texttt{abigen}) e pelo protocolo JSON-RPC. O cliente Go atua como um nó leve ou interage com um nó Geth (\emph{Go-Ethereum}), assinando transações com chaves privadas geridas localmente. A orquestração de todo esse ambiente — incluindo nós de \emph{blockchain} privados, servidores de aplicação e bancos de dados — continua sendo gerida via Docker Compose, essencial para garantir a reprodutibilidade dos testes de consenso e mineração em ambiente laboratorial.

A Tabela~\ref{tab:fund-teorica-pilares-new} atualiza os pilares teóricos do projeto, refletindo a substituição do anel de tokens pela tecnologia de registros distribuídos.

\begin{table}[htbp]
  \centering
  \caption{Pilares teóricos atualizados para a arquitetura baseada em Blockchain.}
  \label{tab:fund-teorica-pilares-new}
  \begin{tabular}{@{} p{0.22\linewidth} p{0.34\linewidth} p{0.34\linewidth} @{}}
    \toprule
    \textbf{Pilar} & \textbf{Referência Principal} & \textbf{Aplicação no Projeto} \\
    \midrule
    Blockchain Architecture & Wood~\cite{wood2014}, Zheng \emph{et al.}~\cite{zheng2017} & Livro-razão imutável para persistência do estado global (estoque e inventário). \\
    Smart Contracts & Szabo~\cite{szabo1997}, Antonopoulos~\cite{antonopoulos2018} & Lógica autônoma para distribuição de pacotes e validação de trocas (\texttt{PackRegistry}). \\
    Arquitetura Híbrida & Xu \emph{et al.}~\cite{xu2019} & Separação entre jogabilidade (Off-chain) e liquidação de ativos (On-chain). \\
    RESTful APIs & Fielding~\cite{fielding2000} & Comunicação síncrona entre servidores para \emph{matchmaking}. \\
    Publish-Subscribe & Eugster \emph{et al.}~\cite{eugster2003} & Desacoplamento de mensagens de eventos para clientes conectados. \\
    Concorrência em Go & The Go Authors~\cite{go-tour-concurrency} & Gerenciamento de múltiplas conexões TCP e listeners de eventos da \emph{chain}. \\
    \bottomrule
  \end{tabular}
\end{table}

A \textbf{Figura~\ref{fig:blockchain-arch}} ilustra a nova topologia, onde os servidores de jogo atuam como clientes da rede Ethereum, submetendo transações para o contrato inteligente compartilhado.

% Sugestão de inserção de figura (se houver)
\begin{figure}[ht]
  \centering
  \includegraphics[width=0.95\linewidth]{tec502-modelo-latex-relatorio-pbl/figuras/arquitetura.png}
  \caption{Arquitetura híbrida: Servidores de Jogo interagindo com Smart Contracts na rede Ethereum.}
  \label{fig:blockchain-arch}
\end{figure}
